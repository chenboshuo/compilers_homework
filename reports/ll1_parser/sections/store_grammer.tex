\documentclass[../report]{subfiles}
\graphicspath{{figures/}{../figures/}}

% \renewcommand{\Comment}{$\hspace*{-0.075em}\rhd$ }
\begin{document}
接下来我们约定:
文法的输入需要遵守 \LaTeX 语法并保证符号之间用空格隔开,
$\epsilon$ 表示空串,
一组非终结符需要加粗。
对于用户来说,这样的输入输出较为美观,
对于程序来说, 可以直接根据空格拆分文法的每一个符号。

根据输入约定,构造数据结构$\id{rules}$,
可以通过Python字典实现,
对于每一个字典项,
用列表存储文法的条目,
每个条目是一个列表,
作为符号序列,
即$\id{Dict[str,List[List[str]]]}$,
该字典的键(key) 为非终结符集合,
同时记录非终结符集合$\id{termials}$,
便于接下来算法处理,
也要记录开始符号,便于接下来使用

例如,对于
\begin{equation}
  \begin{array}{ll}
		E & \to T E' \\
		E' & \to + T E |\, \epsilon \\
		T & \to F T' \\
		T' & \to *F T' |\, \epsilon \\
		F & \to ( E ) |\, \textbf{id}\\
  \end{array}
  \label{eq:test}
\end{equation}

按照输入约定,输入一个列表$\id{g}$表示
文法 (\ref{eq:test})
\begin{lstlisting}[language=python]
g = [r"E \to T E'",
     r"E' \to + T E' | \epsilon ",
     r"T \to F T'",
     r"T' \to * F T' | \epsilon ",
     r"F \to ( E ) | \textbf{id}"]
\end{lstlisting}

在python中存储结构如下
\begin{lstlisting}[language=python]
defaultdict(list,
            {'E': [['T', "E'"]],
             "E'": [['+', 'T', "E'"], ['\\epsilon']],
             'T': [['F', "T'"]],
             "T'": [['*', 'F', "T'"], ['\\epsilon']],
             'F': [['(', 'E', ')'], ['\\textbf{id}']]})
\end{lstlisting}


更多详细的说明可以查阅本项目的API说明文档
\url{https://compilers-homework.readthedocs.io/en/latest/ll1_parser/api/}

\end{document}
