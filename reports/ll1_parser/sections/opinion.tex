\documentclass[../report]{subfiles}
\graphicspath{{figures/}{../figures/}}

% \renewcommand{\Comment}{$\hspace*{-0.075em}\rhd$ }
\begin{document}
这次报告网上现有的资料较少,
搜出的代码可读性不高,
所以本文算法设计根据编译原理
\upcite{alfred_v_aho_compilers_2006}
的描述直接做出,
可能算法不够精炼,
但是实现的过程感受到该书描述的严谨细致,
通过语言描述可以直接构造相应的算法和数据结构。

代码实现是通过我对规则的理解给出,
实现的过程中也在不断改进美化代码和简化逻辑,
这样的过程增加了我程序设计的经验和技巧,
利于接下来进一步学习。
另外代码的许多部分不够简洁,
如$\first(X)$ 和 $\first(\alpha)$ 的算法有逻辑重叠,
希望在有时间的时候对代码进行优化。


通过这些设计研究,
我了解了计算机设计的抽象问题,
分析问题,
解决问题的方法和步骤,
这些思考和实现有利于接下来的程序设计和罗辑思维。

\end{document}
