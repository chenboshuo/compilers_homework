\documentclass[../report]{subfiles}
\graphicspath{{figures/}{../figures/}}

% \renewcommand{\Comment}{$\hspace*{-0.075em}\rhd$ }
\begin{document}

程序$\func{first}(X)$集合规则如下

\begin{enumerate}
  \item 如果$X$为终结符,则 $\first (X) = \{ X \}$
  \item 若$X$ 为非终结符,$X \to Y_1, Y_2, \dots, Y_k (k \ge 1), Y_1 Y_2 \dots Y_{i-1} \xRightarrow{*}\epsilon$,即$\epsilon \in \first(Y_j), j=1,2,\dots i-1$, 将$\first (X_i)$中所有元素加入$\first (X)$
  \item 对于$X \to \epsilon$, 直接将$\epsilon$加入$\first (X) $
\end{enumerate}

根据对应规则和数据结构设计
\cref{alg:create_first}

\begin{algorithm}[H]
  \caption{生成终结符的first集合并存储}
  \begin{codebox}
    \Procname{create\_first()}
      \li \textbf{queue} \id{nonterminals\_rules} \Comment[8] 存储暂时没法处理的终结符
      \li \For left, rule \textbf{in} \id{rules}:
        \Comment[7.5] 迭代每一条规则
      \Then
        \li $x \gets \id{rule[0]}$ \Comment $x$ 为产生式左边第一个符号
        \li \If $ x \in$ \id{terminals}:
          \Comment[5.7] 产生式的第一个符号是终结符
        \Then
          \li \first(left) $\gets \first (\id{left}) \cup \{ x \}$ \Comment 将终结符加入左部的\first 集合
        % \End
        \li \Else \If $ x = \epsilon$:
        % \Then
          \li contain\_empty $\gets contain_empty \cup \{ x \}$
        % \End
        \li \Else: \Comment[2] 终结符稍后处理
        % \Then
          \li
            \id{nonterminals\_rules}
              .\func{enque}((left,rule))
                \Comment[2] 将对应的信息入队
        \End
      \End
      \li \While \id{nonterminals\_rules}: \Comment 若队不空
      \Then
        \li $\id{left,rule} \gets \id{nonterminals}.\func{dequeue}()$
        \li $x \gets \id{rule[0]}$
        \li \If $x \in \id{firsts}$: \Comment firsts 为当前有first集合的非终结符的集合,
        \Then
          \zi \Comment[5.5] 也就是说,他们已经有非终结符在first 集合中
          \li \first (left) $\gets \first (\id{left}) \cup \first (x)$
        % \End
        \li \Else :
        % \Then
          \li
            \id{nonterminals\_rules}
              .\func{enque}$\big((\id{left,rule})\big)$
        \End
        \li \If $x \in contain\_empty$: \Comment $x$ 可能包含空串
        \Then
          \li \If rule[1]: \Comment 有其他的符号
          \Then
            \li nonterminals.enque$\big((\id{left, rule[1:]})\big)$ \Comment 将后续字符加入处理队列
          \End
        \End
      \End
  \end{codebox}
  \label{alg:create_first}
\end{algorithm}

\cref{alg:create_first} 的对应python实现见
\url{https://compilers-homework.readthedocs.io/en/latest/_modules/LL1Parser/#LL1Parser.create_first}

类似的,可以求表达式的文法,
我们可以利用
\cref{alg:create_first}
的结果。


% \cref{alg:first}
\begin{algorithm}[H]
  \caption{求 $first(\alpha)$}
  \begin{codebox}
    \Procname{get\_first($\alpha$)}
      \li $x \gets \alpha[0]$ \Comment $x$ 为表达式第一个字符
      \li \If $x \in \id{terminals}$ \textbf{or} $x = \epsilon$:
      \Then
        \li \Return $\{ x \}$
      \End
      \li $s \gets \first(x)$
      \li \While $\epsilon \in s$:
      \Then
        \li $n \gets  get\_first(\alpha[1:])$
          \Comment 递归的求出接下来的表达式的\first 集合
        \li $s \gets s \cup n$
      \End

      \li \Return $s$ 
  \end{codebox}
  \label{alg:first}
\end{algorithm}

这部分代码在\url{https://compilers-homework.readthedocs.io/en/latest/_modules/LL1Parser/#LL1Parser.get_first}中实现


\end{document}
