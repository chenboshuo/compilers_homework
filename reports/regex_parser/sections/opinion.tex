\documentclass[../report]{subfiles}
\graphicspath{{figures/}{../figures/}}

% \renewcommand{\Comment}{$\hspace*{-0.075em}\rhd$ }
\begin{document}
  这部分实验是一组算法,
  非常锻炼逻辑能力和设计能力,
  对数据类型的封装可以应用许多面相对象技术,
  如实现运算时,
  可以使用一个对象的方法更新,
  也可以使用静态方法,
  基本的自动机类型可以由工厂模式的类方法构造,
  这一些对以后的程序设计很有帮助。

  算法方面,
  文字的分析很好的展现了递归的简单优雅,
  利用递归可以快速构造代码,
  且符合思考的逻辑,
  但是如果想进一步优化算法,
  需要利用栈来做进一步分析。

  这一类算法的调试,
  分析都是从小的部分自顶而上构造,
  自顶而下的思考分析,
  锻炼了逻辑思维。

  对于网上的算法,
  需要自己的理解分析
  对于网上的代码
  需要改造为自己理解的形式,
  直接不假思索的复制,
  沉迷于低质量的信息,
  在我的实验过程中看到了一些经典的讲解
  \upcite{recursive-descent}
  和一些完整的实现
  \upcite{automata-from-regex}
  这一些都是参考学习的优质资源,
  比许多明显是别的学生作业的资源优质许多,
  这些可以对照我们学过的理论
  \upcite{alfred_v_aho_compilers_2006}
  对编译系统有更深的了解,
  进而熟悉一套解析的逻辑,
  增强自己的能力

\end{document}
