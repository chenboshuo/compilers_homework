\documentclass[../report]{subfiles}
\graphicspath{{figures/}{../figures/}}

% \renewcommand{\Comment}{$\hspace*{-0.075em}\rhd$ }
\begin{document}
正则表达式的结构可以由以下解析式组成

\begin{lstlisting}[numbers=none]
<regex>   ::= <term> '|' <regex>
            |  <term>

<term>    ::= { <factor> }

<factor>  ::= <base> { '*' }

<base>    ::= <char>
            |  '\' <char>
            |  '(' <regex> ')'
\end{lstlisting}

这些算法可以用递归实现,
我们首先完成以下函数
\begin{enumerate}
  \item peek() 查看模式串中下一个字符
  \item eat($c$) 检查下一个字符是否为$c$,
    如果是,将模式串中去掉该元素,
    即返回第一个字符之后的元素
  \item next() 返回下一个元素,并在模式串中去掉该元素
\end{enumerate}

我们从最简单的部分开始,
首先通过%
\cref{alg:base}
检测base,
处理转义符,括号和普通单个字母

\begin{algorithm}[H]
  \caption{base 部分的解析}
  \begin{codebox}
    \Procname{base()}
      \li \If peek() $=$ '(':  \Comment[2] 匹配括号
      \Then
        \li eat('(') \Comment[3.5] 处理掉'('
        \li r $\gets$ regex() \Comment[1.5] 这一部分为另一个正则表达式,在\cref{alg:regex}实现
        \li eat(')')
        \li \Return r \Comment[3] 此时括号内的表达式的NFA是解析结果
      \End
      \li \Else \If peek() $=$ '\textbackslash': \Comment[2] 处理转义符之后的字符
      \Then
        \li eat(‘\textbackslash’)
        \li esc $\gets$ next() \Comment[3.5] 获得转义符之后的一个字符
        \li \Return basicConstruct(esc) \Comment[1] 创建只含有$\id{esc}$的NFA
      \End
      \li \Else:
      \Then
        \li \Return basicConstruct(next()) \Comment[1] 只有一种字母的情况
      \End
  \end{codebox}
  \label{alg:base}
\end{algorithm}

factor 部分由
base 和若干个*组成,
我们对\cref{alg:base}
的结果进行star运算(\cref{alg:star})
具体过程如%
\cref{alg:factor}

\begin{algorithm}[H]
  \caption{factor 部分的解析}
  \begin{codebox}
    \Procname{factor()}
      \li base $\gets$ base() \Comment[1] 从\cref{alg:base} 中得到*之前部分的NFA
      \li \While parttern \kw{and} peek() $=$ '*':
        \Comment[1] 处理所有'*'字符,并进行相应运算
      \Then
        \li eat('*')
        \li star(base)
          \Comment[1]  对base进行star运算(\cref{alg:star})
      \End
      \li \Return base
  \end{codebox}
  \label{alg:factor}
\end{algorithm}

term 部分由若干个 factor 组成,
他们之间用 concatenation 运算链接
(\cref{alg:concatenation})
具体实现如%
\cref{alg:term}

\begin{algorithm}[H]
  \caption{term 部分的解析}
  \begin{codebox}
    \Procname{term()}
      \li term $\gets$ basicConstruct($\epsilon$)
        \Comment[1] 只有空串转换的NFA
      \li \While pattern \kw{and} peek() $\neq$ ')' \kw{and} peek() $\neq$ '*': \Comment 下一个字符不能是需要运算符号
      \Then
        \li f $\gets$ factor() \Comment 读取下一个term(\cref{alg:factor})
        \li term $\gets$  concatenation(term, f)
          \Comment 更新 term
      \End
      \li \Return term
  \end{codebox}
  \label{alg:term}
\end{algorithm}

最终,我们可以将一个正则表达式分解为term
或者term '|' regex的形式
我们对相应NFA union 运算
(\cref{alg:union})
然后得到最终的NFA
具体细节见%
\cref{alg:regex}

\begin{algorithm}[H]
  \caption{regex 部分的解析}
  \begin{codebox}
    \Procname{regex()}
      \li term $\gets$ term() \Comment 第一部分为term部分
      \li \If pattern \kw{and} peek() $=$ '|':
      \Then
        \li regex $\gets$ regex() \Comment 下一部分是另一个正则表达式,用自身解析
        \li \Return union(term,regex)
      \End
      \Else:
      \Then
        \li \Return term
      \End
  \end{codebox}
  \label{alg:regex}
\end{algorithm}

\end{document}
