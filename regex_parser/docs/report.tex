\documentclass[pdfCover]{myreport} % 需要pdf封面
% \documentclass{myreport}
\title{测试myreport}
\author{陈伯硕}
\date{\today}


\usepackage{subfiles}
\begin{document}
\maketitle
\pagestyle{empty}
% Generate the Table of Contents if it's needed.
% \tableofcontents
% \newpage

\setcounter{section}{3}
\section{算法设计}
  \subsection{自动机的数据结构与算法}
    \subfile{sections/create_automata}
  \subsection{正则表达式解析}
    \subfile{sections/parse}

\section{实验步骤与结果}
  \subsection{代码测试自动机的基本构造}
    \subfile{sections/test_automaton}
  \subsection{测试自动机的方法和运算}
    \subfile{sections/test_operation}
  \subsection{测试正则表达式解析}
    \subfile{sections/test_parser}




\section{实验总结}
  \subfile{sections/opinion}

\bibliography{reference}


\begin{appendices}
  \section{依赖的安装}
    本文代码的Atutomata.draw()方法依赖
    \lstinputlisting{../src/requirements.txt}
    可以通过以下命令在清华镜像网站安装全部
    \begin{lstlisting}
    pip install -i http://mirrors.aliyun.com/pypi/simple \
    --trusted-host mirrors.aliyun.com/pypi/simple/ \
    -r requirements.txt
    \end{lstlisting}
    如果在程序中不使用画图功能,
    可以忽略安装

  \section{自动机的相关方法与测试源代码}
    \lstinputlisting[language=python]{../src/Automata.py}
  \section{正则表达式的解析}
    \lstinputlisting[language=python]{../src/RegexParser.py}
\end{appendices}
\end{document}
