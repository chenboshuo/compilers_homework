\documentclass[../main.tex]{subfiles}
\graphicspath{{figures/}{../figures/}}

\begin{document}
\subsubsection{自动机的数据结构}
  自动机由元组$(\Sigma,S,S_0,F,f)$
  组成,其中
  \begin{enumerate}
    \item $\Sigma$ 为字符的集合,空串$\epsilon \notin \Sigma$
    \item $S$为状态集合
    \item $S_0 \in S$为初始态
    \item $F \subset S$是终态的集合
    \item $f:S \times (\Sigma \cup \{ \epsilon\}) \to S$为状态转换函数,
      在程序中可以使用哈希表
      $\func{hash}(S_i,S_{j_k}) = C_{ij_k}$,
      $\forall c_{ij_k} \in C_{ij_k}$,
      使得
      $f(S_i,c_{ij_k}) = S_{j_k}$
      在python中,
      hash函数由字典数$\id{dict}$据类型直接实现,
      对于两个变量的哈希表,
      可以用字典嵌套字典实现
  \end{enumerate}



\end{document}
